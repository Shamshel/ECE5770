% Use only LaTeX2e, calling the article.cls class and 12-point type.

\documentclass[12pt]{article}

% Users of the {thebibliography} environment or BibTeX should use the
% scicite.sty package, downloadable from *Science* at
% www.sciencemag.org/about/authors/prep/TeX_help/ .
% This package should properly format in-text
% reference calls and reference-list numbers.

\usepackage{scicite}

% Use times if you have the font installed; otherwise, comment out the
% following line.

\usepackage{times}

% The preamble here sets up a lot of new/revised commands and
% environments.  It's annoying, but please do *not* try to strip these
% out into a separate .sty file (which could lead to the loss of some
% information when we convert the file to other formats).  Instead, keep
% them in the preamble of your main LaTeX source file.


% The following parameters seem to provide a reasonable page setup.

\topmargin 0.0cm
\oddsidemargin 0.2cm
\textwidth 16cm 
\textheight 21cm
\footskip 1.0cm


%The next command sets up an environment for the abstract to your paper.

\newenvironment{sciabstract}{%
\begin{quote} \bf}
{\end{quote}}


% If your reference list includes text notes as well as references,
% include the following line; otherwise, comment it out.

\renewcommand\refname{References and Notes}

% The following lines set up an environment for the last note in the
% reference list, which commonly includes acknowledgments of funding,
% help, etc.  It's intended for users of BibTeX or the {thebibliography}
% environment.  Users who are hand-coding their references at the end
% using a list environment such as {enumerate} can simply add another
% item at the end, and it will be numbered automatically.

\newcounter{lastnote}
\newenvironment{scilastnote}{%
\setcounter{lastnote}{\value{enumiv}}%
\addtocounter{lastnote}{+1}%
\begin{list}%
{\arabic{lastnote}.}
{\setlength{\leftmargin}{.22in}}
{\setlength{\labelsep}{.5em}}}
{\end{list}}


% Include your paper's title here

\title{Remote Rover Training System\\Customer Requirements} 


% Place the author information here.  Please hand-code the contact
% information and notecalls; do *not* use \footnote commands.  Let the
% author contact information appear immediately below the author names
% as shown.  We would also prefer that you don't change the type-size
% settings shown here.

\author{Cody Herndon}

% Include the date command, but leave its argument blank.

\date{}



%%%%%%%%%%%%%%%%% END OF PREAMBLE %%%%%%%%%%%%%%%%



\begin{document} 

% Double-space the manuscript.

\baselineskip24pt

% Make the title.

\maketitle 



% Place your abstract within the special {sciabstract} environment.

\begin{sciabstract}
  This document outlines customer requirements for the Remote Rover Training System.  This system is intended to provide scientists and engineers with valuable experience developing and using robust remote robotics platforms and operating systems in a low-cost, low-risk enviroment.
\end{sciabstract}



% In setting up this template for *Science* papers, we've used both
% the \section* command and the \paragraph* command for topical
% divisions.  Which you use will of course depend on the type of paper
% you're writing.  Review Articles tend to have displayed headings, for
% which \section* is more appropriate; Research Articles, when they have
% formal topical divisions at all, tend to signal them with bold text
% that runs into the paragraph, for which \paragraph* is the right
% choice.  Either way, use the asterisk (*) modifier, as shown, to
% suppress numbering.

\section{Introduction}
The Remote Rover Training System (hereon referred to as ``the System'') is a mechanically simple mobile robot intended to provide a training platform for aspiring engineers and scientists who may use such robotics in the course of their duties.  The System is intended to provide a gentile and easy introduction to developing real-time operating systems on a remote mobile robot which is easily replaceable and robust enough to avoid being damaged by faulty programming.

\section{Physical Requirements}
These requirements outline restrictions on the System regarding size, weight, accessibility, and mobility.
\subsection{The System shall weigh no more than can be carried by a single person.}
\subsection{The System shall be capable of fitting in a person's arms comfortably.}
\subsection{The system shall expose all components and circuitry to the user with the removal of at most one access panel.}
\subsection{The system shall be capable of easily traversing a room with a carpet of no more than 1/2 inch thickness.}

\section{Electrical Requirements}
These requirements outline restrictions on the System regarding operational duration, range, and sensor capabilities.

\subsection{Power and Operational Requirements}
\subsubsection{The System shall be battery powered.}
\subsubsection{The System shall utilize a battery which is rechargeable.}
\subsubsection{The System shall have enough stored energy to operate without activating its motors for at least an hour without recharging.}
\subsubsection{The System shall have enough stored energy to traverse a room without recharging.}

\subsection{Transmission and Communication Requirements}
\subsubsection{The System shall be capable of receiving commands remotely.}
\subsubsection{The System shall be capable of sending sensor data remotely.}
\subsubsection{The System shall be capable of receiving software updates remotely.}

\subsection{Sensor Requirements}
\subsubsection{The System shall be capable of determining the distance travelled from its origin.}
\subsubsection{The System shall be capable of determining the remaining battery power available.}
\subsubsection{The System shall be capable of determining the power dissipated through the drive motors.}

\section{Mechanical Requirements}
These requirements outline restrictions on mechanical operations the System shall be capable of performing.

\subsection{Motion Requirements}
\subsubsection{The System shall be capable of travelling forward and reverse.}
\subsubsection{The System shall be capable of rotating clockwise and counterclockwise without travelling forward or reverse.}

\end{document}
